\section{Fazit}

Im ersten Teil der Arbeit wurde mit der Existenz einer perfekten Eliminations-Reihenfolge eine äquivalente Aussage zu der Chordalität des Graphen dargestellt und bewiesen. Anschließend haben wir mit der Kardinalitätssuche einen Algorithmus gezeigt, der dazu verwendet werden kann, eine perfekte Eliminations-Reihenfolge -- falls möglich -- zu finden und damit chordale Graphen zu erkennen. Daraufhin wurden, neben der Standard-Implementierung, zwei platzeffizientere Varianten vorgestellt. Zum Abschluss haben wir außerdem mit der Erhaltung von \( 0 \)-Einträgen in dünnbesetzten Matrizen mithilfe von perfekten Elimination-Reihenfolgen beim Gauß-Algorithmus eine praktische Anwendung der vorgestellten theoretischen Grundlagen prä\-sen\-tiert.
