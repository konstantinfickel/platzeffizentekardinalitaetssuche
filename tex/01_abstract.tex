Sei \( G \) ein ungerichteter Graph. \( G \) ist chordal, wenn jeder Kreis mit mindestens vier Knoten in \( G \) eine Sehne besitzt. In dieser Seminararbeit, die hauptsächlich auf der Arbeit \glqq{}Space-Efficient Algorithms for Maximum Cardinality Search, Stack BFS, Queue BFS and Applications\grqq{} von Sankardeep Chakraborty und Srinivasa Rao Satti (\cite{sankardeep})  sowie auf Kapitel 4 des Buches \glqq{}Algorithmic graph theory and perfect graphs\grqq{} von Martin Charles Golumbic (\cite{golumbic}) basiert, wird zuerst eine Möglichkeit, chordale Graphen zu erkennen, vorgestellt. Anschließend werden für einen Teilschritt davon verschiedene Zeit/Platz-Tradeoffs erklärt.
